% please run using
% lualatex --shell-escape proof_outline.tex

\documentclass[sigplan,screen]{acmart}
\usepackage[utf8x]{inputenc}
\usepackage[T1]{fontenc}
\usepackage{ulem}
%\usepackage{amsmath}

\usepackage[textsize=tiny]{todonotes}

\pdfoutput=1

\usepackage{minted}
\setminted{encoding=utf-8}
\usepackage{fontspec}
%\setmainfont{dejavu-serif}
\setmonofont[Scale=0.833]{dejavu-sans}
%\setmainfont{FreeSerif}
%\setmonofont{FreeMono}
\usepackage{xcolor, graphics}
\definecolor{mygray}{rgb}{0.97,0.97,0.97}
\definecolor{LightCyan}{rgb}{0.8784,1,1}
\newcommand{\leanline}[1]{\mintinline[breaklines, breakafter=., breaklines]{Lean}{#1}}%\textsuperscript{\textcolor{gray}{\tiny(m)}}}
\newcommand{\leanlinecolor}{\mintinline[breaklines, breakafter=., breaklines, bgcolor=LightCyan]{Lean}}
%\usepackage{refcheck}

% Definitions and packages for Lean code
\usepackage{color}
\definecolor{keywordcolor}{rgb}{0.7, 0.1, 0.1}   % red
\definecolor{commentcolor}{rgb}{0.4, 0.4, 0.4}   % grey
\definecolor{symbolcolor}{rgb}{0.0, 0.1, 0.6}    % blue
\definecolor{sortcolor}{rgb}{0.1, 0.5, 0.1}      % green

\usepackage{enumitem}
\setlist[enumerate]{%
align=left, itemsep=0pt, leftmargin=0pt, labelindent=0pt, listparindent=0pt, labelwidth=0pt, itemindent=!
}
%\theoremstyle{plain}
\newtheorem{proposition}{Proposition}[section]
%\newtheorem{theorem}[proposition]{Theorem}
\newtheorem{theorem}{Theorem}
\newtheorem{lemma}[proposition]{Lemma}
\newtheorem{corollary}[proposition]{Korollar}
\newtheorem{definition}[proposition]{Definition}
\newtheorem{remark}[proposition]{Remark}

\author{Rémy Degenne}
\affiliation{%
   \institution{Univ. Lille, Inria, CNRS, Centrale Lille, UMR 9189-CRIStAL}
   \city{Lille}
   \country{France}}
\email{remy.degenne@inria.fr}

\author{Peter Pfaffelhuber}
\affiliation{%
   \institution{University of Freiburg}
   \city{Freiburg}
   \country{Germany}}
\email{p.p@stochastik.uni-freiburg.de}

\title{Formalizing the Kolmogorov extension Theorem in Lean}

\keywords{probability, measure theory, Lean, formal mathematics, proof assistant, mathlib}

% Tool to generate below available at http://dl.acm.org/ccs.cfm.
%\begin{CCSXML}
%<ccs2012>
%<concept>
%<concept_id>10002950.10003648</concept_id>
%<concept_desc>Mathematics of computing~Probability and statistics</concept_desc>
%<concept_significance>500</concept_significance>
%</concept>
%<concept>
%<concept_id>10003752.10003790.10002990</concept_id>
%<concept_desc>Theory of computation~Logic and verification</concept_desc>
%<concept_significance>500</concept_significance>
%</concept>
%</ccs2012>
%\end{CCSXML}

%\ccsdesc[500]{Mathematics of computing~Probability and statistics}
%\ccsdesc[500]{Theory of computation~Logic and verification}

%%% If you see 'ACMUNKNOWN' in the 'setcopyright' statement below,
%%% please first submit your publishing-rights agreement with ACM (follow link on submission page).
%%% Then please reload our instructions page and copy-and-paste the NEW commands into your article.
%%% Please contact us in case of questions; allow up to 10 min for the system to propagate the information.
%%%
%%% The following is specific to CPP '23 and the paper
%%% 'A Formalization of Doob's Martingale Convergence Theorems in Mathlib'
%%% by Kexing Ying and Rémy Degenne.
%%%
%\setcopyright{licensedothergov}
%\acmPrice{15.00}
%\acmDOI{10.1145/3573105.3575675}
%\acmYear{2023}
%\copyrightyear{2023}
%\acmSubmissionID{poplws23cppmain-p22-p}
%\acmISBN{979-8-4007-0026-2/23/01}
%\acmConference[CPP '23]{Proceedings of the 13th ACM SIGPLAN International Conference on Certified Programs and Proofs}{January 13--14, 2024}{London, UK}
%\acmBooktitle{Proceedings of the 13th ACM SIGPLAN International Conference on Certified Programs and Proofs (CPP '24), January 13--14, 2024, London, UK}
%\received{2023-09-18}
%\received[accepted]{??}

%\usepackage{environ}
%\NewEnviron{killcontents}{}
%\let\proof\killcontents
%\let\endproof\endkillcontents

\begin{document}

\begin{abstract}
  We present a formalization in Lean of the Kolmogorov extension theorem,
  which is a main building block in the construction of the (distribution of)
  stochastic processes with arbitrary index sets. Our
  approach is based on mathlib, the mathematical library for Lean. On
  our way, we provide a formalization of the Caratheodory extension
  theorem, which allows to extend a set function on a semi-ring to a
  proper measure on the $\sigma$-algebra generated by the semi-ring. A
  slight generalization of the classical Kolmogorov extension theorem
  allows us to construct the (distribution of) a stochastic process on
  a complete and separable pseudo-metric space rather than a metric
  space.
\end{abstract}

\maketitle

\section{Introduction}
One of the main building blocks of modern probability theory are stochastic processes, which are usually defined as any collection of random variables -- $(X_t)_{t\in \iota}$ with $X_t$ taking values in some $\alpha_t$ for all $t\in\iota$ -- defined on some joint probability space $(\Omega, \mathcal A, \mathbf P)$.  (As usual, we will refer to $\iota$ as the index set of times.) In order to study such processes, it is fundamental to talk about their joint distribution, i.e.\ a probability distribution on the product set $\prod_{t\in\iota} \alpha_t$. The usual approach to construct (the distribution of) a stochastic process works as follows: describe properties of the distribution of the stochastic process $P_J$ at some arbitrary but finite number of times $J = \{t_1,...,t_n\} \subseteq \iota$.
The resulting family of probability measures $(P_J)_{J \subseteq \iota \text{ finite}}$ has to be {\em projective} in the sense that the projection of $P_J$ to $H\subseteq J$ has to be equal to $P_H$. In other words, when describing the distribution of the stochastic process at all times in $J$, and then forgetting all properties for times in $J\setminus H$, results in the description of properties at times in $H$. One may then ask if this already gives a complete description of the process for all times.
For uncountable $\iota$, e.g. $\iota = [0,\infty)$, one is tempted to be pessimistic at first sight since measures -- which describe the distribution of the stochastic process -- usually only deal well with a countable number of measurable events. However, it is the achievement of Kolmogorov that the finite-dimensional distributions in fact provide a unique description of the distribution of a stochastic process, as long as the underlying family of state spaces $(\alpha_t)_{t\in\iota}$ is nice enough (Polish, i.e.\ a separable topological space which can be metrized by a complete metric, for example).
Assuming a family of topological spaces $(\alpha_t)_{t\in\iota}$, this distribution is a measure on the product-$\sigma$-field $\mathcal F :=\bigotimes_{t\in\iota} \mathcal B(\alpha_t)$ (where $\mathcal B(\alpha_t)$ is the Borel $\sigma$-algebra on $\alpha_t$). Here, $\mathcal F$ is generated by finite projections and hence any element of $\mathcal F$ may only depend on at most countably many $t\in \iota$, making this a rather coarse $\sigma$-algebra. (In particular, note that this is not the Borel $\sigma$-algebra of the product topology for infinite $\iota$.) The result responsible for this insight is usually denoted the {\em Kolmogorov extension theorem}, formulated in \cite{kolmogoroff1933grundbegriffe}. We note that a version of the extension result was proved by Daniell in the 1930s, but this paper was not acknowledged by the probabilists of that time \cite{aldrich2007but}. Due to his contribution, the theorem is often called the {\em Daniell-Kolmogorov extension}. 
  
The goal of our contribution is a formalization of the proof of that theorem in {\tt Lean}. We are using \leanline{Lean4} and its mathematical library \leanline{mathlib} \cite{mathlib2020lean}\footnote{All references to results in \leanline{mathlib} are accurate for commit {\tt 537e7f030d6df737d728c7c002fb1928f4615afa}, August 22, 2023.}.
The Kolmogorov extension theorem is on the interface between measure theory and probability theory. Here, we rely on a decent amount of formalized mathematics in the measure-theory part of {\tt mathlib} (outer measures, above all), while not requiring any specific previous formalization of probability theory. (In fact, most of our results are formulated in terms of finite rather than probability measures.)

At first sight, it might be surprising that martingales, a certain class of stochastic process, have already been formalized in Lean \cite{ying2023formalization}, although the Kolmogorov extension theorem (on the existence of stochastic processes) is only available by our contribution. Note, however, that a martingale $(X_t)_{t\in \iota}$ is defined as a family of random variables (satisfying some properties) on a fixed probability space $(\Omega, \mathcal F, \mathbf P)$, while the Kolmogorov extension theorem is on the construction of a probability (or finite) measure $\mathbf P$, on which we can define random variables (with certain properties). So, our work complements \cite{ying2023formalization} in the sense that probability spaces $(\Omega, \mathcal F, \mathbf P)$ on which you can define a martingale exist. {\color{blue}As an application of our implementation, we construct a probability measure on an infinite product space where all coordinates are independent; see Section~\ref{ss:concl}.}

The main steps in our construction, which were previously missing in {\tt mathlib}, are (more mathematical details are below, and in standard textbooks on Probability Theory, e.g.\ \cite{Klenke2013, Kallenberg2020}):
\begin{enumerate}
\item a formalization of (the set-system of) semi-rings (see
  Definition~\ref{def:semi});
\item a definition of additive contents over semi-rings (see Definition~\ref{def:content});
\item a proof that a single probability measure on a Polish space is inner regular with respect to compact sets (see Definition~\ref{def:innerreg}; in fact, we proved a slight generalization using pseudo-metric spaces, found in Lemma~\ref{L:relcoPol} below); 
\item the classical Carathéodory extension theorem, providing us with a candidate for the measure which we want to construct (see Theorem~\ref{T:masseind}); 
\item the Kolmogorov extension theorem, as based on the previous steps (See Theorem~\ref{T1}).
\end{enumerate}
The Kolmogorov extension theorem has been previously formalized in Isabelle/HOL \cite{Immler2012}. This formalization only works on Polish spaces (rather than on spaces where every finite measure is inner regular with respect to compact sets, see below), and only in the case where all $\alpha_i$'s are identical. %In addition, Isabelle/HOL works with simple type theory, while Lean is an implementation of dependent type theory. 

\subsubsection*{Possible future work}
\sloppy Let us describe some future projects extending {\tt mathlib} which become possible by our contribution.

\noindent
{\em Instances of stochastic processes:} An obvious application of Kolmogorov's extension theorem is the construction of basic stochastic processes like Markov chains \cite[Section~11]{Kallenberg2020}, the Poisson process \cite[Section~13]{Kallenberg2020} and Brownian motion \cite[Section~14]{Kallenberg2020}.
Other Gaussian processes -- indexed by $[0, \infty)$ or $\mathbb R$ -- might as well be constructed the same way. In addition, fields like the Gaussien free field indexed by $\mathbb R^d$ \cite{sheffield2007gaussian} can also be given.
For these tasks, we would have to define the finite-dimensional distributions (Poisson and normally distributed, respectively), and apply the extension theorem. This task requires the formalization of multi-dimensional Poisson and normal distributions, which -- in textbooks -- is usually done using characteristic functions. Since these are not yet part of {\tt mathlib}, we postpone this task to the future. 

\noindent
{\em Sample-path properties:} The Kolmogorov extension gives the existence of a distribution of a stochastic process $(X_t)_{t\in\iota}$ with certain properties. Extra work is needed in order to show that -- on the same probability space -- we can as well define a version (i.e.\ another process $(Y_t)_{t\in\iota}$ with $\mathbf P(X_t = Y_t) = 1$ for all $t\in\iota$) which is right-continuous with left limits (for the Poisson process) or continuous (for Brownian motion).
For the former, this follows from some general principles of Markov processes (e.g.\ \cite[Theorem 4.3.6]{EthierKurtz1986}). For the latter, this requires formalization of the Kolmogorov-Chentsov criterion 
\cite[Theorem~4.23]{Kallenberg2020}.

\noindent
{\em Theorem of Ionescu-Tulcea and related results:}
While Kolmogorov's extension Theorem gives a result for arbitrary $\iota$, but requires some properties of the family $(\alpha_t)_{t\in\iota}$, the Theorem of Ionescu-Tulcea \cite[Theorem 8.24]{Kallenberg2020} can only deal with countable $\iota$, but has no restrictions on $(\alpha_t)_{t\in\iota}$. The proof of the latter, however, uses transition kernels (and an inductive construction of a content, which can be extended to a measure) rather than projective families.
However, the structure of the argument is similar and could use some of our results, like the proof of $\sigma$-additivity from continuity at the empty set (see Lemma~\ref{P:stetigmass}).
A formalization would complement our work. In addition, using the Theorem of Ionescu-Tulcea one can define infinitely many independent random variables (arbitrary $\iota$) on a joint probability space \cite[Theorem 8.24]{Kallenberg2020} without requirements on $(\alpha_t)_{t\in\iota}$.
All of these results can be used in order to construct countably many independent random variables, which are frequently used in concrete constructions in probability theory. Examples are (one direction of) the Borel-Cantelli-Lemma \cite[Theorem 4.28]{Kallenberg2020}, Kolmogorov's 0-1-law \cite[Theorem 4.13]{Kallenberg2020}, random walks \cite[Chapter 12]{Kallenberg2020}, branching processes  \cite[Chapter 13]{Kallenberg2020}, and percolation  \cite[Chapter 2.4]{Klenke2013}, to name just a few. While the first two are already formalized in \leanline{mathlib}, it is only clear due to the Kolmorov extension theorem (or the Theorem of Ionesu-Tulcea) that infinitely many independent random variables can indeed be defined on a joint probability space. In this sense, our contribution is also important for giving more sense to already formalised parts of \leanline{mathlib}.

\section{Formalization of the Daniell-Kolmogorov extension}
We start by stating the exact result. We are going to formulate the main result in a modern fashion, as e.g.\ found in Theorem 2.2 of \cite{rao1971projective}, Theorem 7.7.1 of Volume~2 of \cite{bogachev2007measure}, Theorem 15.26 of \cite{guide2006infinite}, or \cite{border1998expository}. Note that these formulations split general assumptions on the underlying space(s) (e.g.\ a metric property) from the property which is needed in the proof (inner regularity with respect to compact sets). Other -- highly readable -- references such as \cite{Billingsley1995} state the extension theorem only in special cases such as $\alpha_t = \mathbb R$ for all $t$. 
%In addition, there are still new proofs found for the general extension theorem \cite{chin2019new}, but they require some results not yet formalized in mathlib (Borel isomorphism of $\mathbb R$ to $\{0,1\}^{\mathbb N}$), so we do not follow this route.  

\subsection{Formulating the result}
Before we can state the result, we begin with some mathematical notions. We start off from metric spaces, but quickly introduce projective families of measures. In all definitions, $\alpha$ will be some set (or type, since {\tt Lean} is a dependently typed language, \leanline{α : Type _}). 

\begin{definition}[Metric and topological spaces]
  \begin{enumerate}
  \item \sloppy A pseudo-metric on $\alpha$ is a symmetric map $r : \alpha
    \times \alpha \to [0,\infty)$ satisfying the triangle inequality,
      i.e.\ $r(x,z) \leq r(x,y) + r(y,z)$ for all $x,y,z\in\alpha$.
      \\ If $r$ also satisfies $r(x,y) = 0$ iff $x=y$, we call it a
      metric.  \\ If $r : \alpha \times \alpha \to [0,\infty]$
      (i.e.\ $r(x,y) = \infty$ is allowed), we call $r$ an extended
      (pseudo-)metric.
    \item Let $r$ be an (extended pseudo-)metric. A sequence $x_1,
      x_2,... \in \alpha$ is called Cauchy if for all $\varepsilon>0$
      there is $N\in\mathbb N$ such that $r(x_n, x_m) < \varepsilon$
      for all $m,n > N$. The (extended pseudo-) metric is called
      complete if every Cauchy-sequency has a limit in
      $\alpha$.
    \item Some\footnote{We denote by $2^\alpha$ the power set of
      $\alpha$, i.e.\ the set of all subsets of $\alpha$.} $\mathcal O
      \subseteq 2^\alpha$ is called a topology, if it satisfies (i)
      $\emptyset, \alpha \in \mathcal O$, (ii) $\mathcal O$ is a
      $\pi$-system, i.e.\ it is stable under finite intersections,
      i.e.\ if $A, B \in \mathcal O$, then $A \cap B \in \mathcal O$,
      and (iii) $\mathcal O$ is stable under arbitrary unions,
      i.e.\ if $A_i \in \mathcal O$ for all $i\in\iota$ and $\iota$ is
      arbitrary, then $\bigcup_{i\in \iota} A_i \in \mathcal O$.
  \end{enumerate}
\end{definition}

\begin{remark}[Metric and topological spaces in {\tt mathlib}] \label{rem:uniform}\mbox{}
  \begin{enumerate}
    \item \sloppy \leanline{mathlib} provides the \leanline{class PseudoEMetricSpace (α : Type u)},
      which comes with an extended pseudo-distance
      \leanline{edist: α → α → ℝ≥0∞} (the notation \leanline{ℝ≥0∞}
      stands for $[0, \infty]$), and the properties
      \leanline{edist_self} (distance to self is zero),
      \leanline{edist_comm} (commutativity) and
      \leanline{edist_triangle} (triangle inequality).
      That class also comes with \leanline{toUniformSpace} and
      \leanline{uniformity_edist}, defining a uniform space from the
      extended pseudo-metric.
    \item A uniform space (\leanline{class UniformSpace (α : Type u)}) does not come with a
      metric, but with a filter on $\alpha \times \alpha$, which
      describes which points in $\alpha$ are near. For example, the
      diagonal of $\alpha \times \alpha$ is a subset of all sets in
      the uniformity; see \cite{james2013topologies} for details. We
      note that a uniform space with a countably generated uniformity
      filter is pseudometrizable, i.e.\ there exists a
      pseudo-metric-space structure that generates the same
      uniformity; see \leanline{UniformSpace.pseudoMetrizableSpace},
      which formalizes a result stated in
      \cite{melikhov2011metrizable}.
    \item \sloppy A topological space is defined in \leanline{mathlib} as \leanline{class TopologicalSpace(α : Type u)},
      which comes with a predicate \leanline{IsOpen}, and properties \leanline{isOpen_univ : IsOpen Set.univ}, \leanline{isOpen_inter} and
      \leanline{isOpen_sUnion}, which are exactly the properties of a
      topological space described above.
      A uniform space defines a topology, given by
      \leanline{∀ (s : Set α), IsOpen s ↔ ∀ (x : α), x ∈ s → {p | p.fst = x → p.snd ∈ s} ∈ uniformity}.
%    \item Any (extended pseudo-)metric on $\alpha$ defines a topology,
%      namely the topology generated by $\mathcal H := \{\{y:
%      r(x,y)<\varepsilon\}: x \in \alpha, \varepsilon \in
%      (0,\infty)\}$. Actually, in \leanline{mathlib}, as we have seen
%      above, every extended pseudo-metric defines a uniform space, and
%      a uniform space is a \leanline{class UniformSpace(α : Type u) extends TopologicalSpace, UniformSpace.Core} (where the
%      uniformity is defined in \leanline{UniformSpace.Core}), which
%      connects the \leanline{uniformity} with the topology using
%      \leanline{∀ (s : Set α), IsOpen s ↔ ∀ (x : α), x ∈ s → {p | p.fst = x → p.snd ∈ s} ∈ uniformity}. In particular, a
%      uniform space defines a topological space, which can be used by
%      the typeclass system.
  \end{enumerate}
\end{remark}

\begin{remark}[Generated topology]
  \begin{enumerate}
  \item The intersection of any number of topologies is again a
    topology. For this reason, if $\mathcal H \subseteq 2^\alpha$, we
    define the topology $\mathcal O := \bigcap_{\mathcal F \supseteq
      \mathcal H \text{ topology}} \mathcal F$; see
    \leanline{TopologicalSpace.generateFrom}. This is called the
    topology generated by $\mathcal H$. If $\mathcal H$ is closed
    under finite intersections, we call $\mathcal H$ a basis for
    $\mathcal O$; see \leanline{TopologicalSpace.IsTopologicalBasis}.
  \item \sloppy We call the topology (generated from an extended
    pseudo-metric) separable (see
    \leanline{TopologicalSpace.SeparableSpace}) if there is a
    countable $s \subseteq \iota$ such that $\inf\{r(x,y) : y \in s\}
    = 0$ for all $x\in\alpha$. (More generally, a
    \leanline{TopologicalSpace α} is a \leanline{SeparableSpace} iff
    \leanline{∃ s, Set.Countable s ∧ Dense s}, where the latter means
    \leanline{∀ (x : α), x ∈ closure s}, i.e.\ the space equals its
    own closure.) \\ If there is a countable basis of the topology, it
    is separable; see
    \leanline{TopologicalSpace.SecondCountableTopology.to_separableSpace}.
  \end{enumerate}
\end{remark}

\noindent
Finally, we can introduce measures. A measure is defined on a
$\sigma$-algebra, which we introduce next.

\begin{definition}[$\sigma$-algebras and measures]\label{def:sigma}
  \begin{enumerate}
  \item 
    We call $\mathcal F \subseteq 2^\alpha$ a $\sigma$-algebra (on
    $\alpha$) if (i) $\emptyset \in \mathcal F$, (ii) $\mathcal F$ is
    stable under complements, i.e.\ $A \in \mathcal F \Rightarrow A^c
    \in \mathcal F$, (iii) $\mathcal F$ is stable under countable
    unions, i.e.\ $A_1, A_2,... \in \mathcal F \Rightarrow
    \bigcup_{n=1}^\infty A_n \in \mathcal F$. We call $(\alpha,
    \mathcal F)$ a measurable space.
  \item For some $\sigma$-algebra $\mathcal F$ on $\alpha$, a function
    $\mu : \mathcal F \to [0, \infty]$ is called a measure, if (i)
    $\mu(\emptyset) = 0$, (ii) $\mu$ is countably additive, i.e.\ for
    $A_1, A_2,... \in \mathcal F$ pairwise disjoint, we have $\mu\Big(
    \bigcup_{n=1}^\infty A_n\Big) = \sum_{n=1}^\infty \mu(A_n)$. We
    call $(\alpha, \mathcal F, \mu)$ a measure space. \\ In addition,
    $\mu$ is called finite if $\mu(\alpha) < \infty$ and a probability
    measure if $\mu(\alpha)=1$.
  \end{enumerate}
\end{definition}

\begin{remark}[Measur(abl)e spaces in \leanline{mathlib}]
  \begin{enumerate}
  \item \sloppy The \leanline{class MeasurableSpace α} is very similar
    to a topological space in \leanline{mathlib}, since it comes with
    \leanline{MeasurableSet' : Set α → Prop},
    \leanline{measurableSet_empty : MeasurableSet' ∅},
    \leanline{measurableSet_compl : (s : Set α) → MeasurableSet' s → MeasurableSet' sᶜ}, and \leanline{measurableSet_iUnion},
    i.e.\ properties (i)-(iii) from Definition~\ref{def:sigma}.1.
  \item We are using outer measures in our construction. In
    \leanline{mathlib}, this is \leanline{structure MeasureTheory.OuterMeasure α} which is a function
    \leanline{measureOf : Set α → ℝ≥0∞} such that \leanline{measureOf ∅ = 0} (i.e.\ the empty set has measure~$0$),
    \leanline{∀ {s₁ s₂ : Set α}, s₁ ⊆ s₂ → measureOf s₁ ≤ measureOf s₂}
    (i.e.\ monotonicity) and \leanline{∀ (s : ℕ → Set α), measureOf (⋃ (i : ℕ), s i) ≤ ∑' (i : ℕ), measureOf (s i)} (which we call
    $\sigma$-subadditivity). \\ Note that an outer measure is defined
    on $2^\alpha$, whereas a measure is only defined on a subset (the
    $\sigma$-algebra of measurable sets).
  \item A measure defined on a type \leanline{α} on which we have a \leanline{MeasurableSpace α} is denoted by the type \leanline{MeasureTheory.Measure α}
    %More precisely\todo{I don't understand why all these details are here. We never use MeasureSpace}: A
    %\leanline{MeasureSpace α} extends a \leanline{MeasurableSpace α}
    %by a \leanline{Measure α} (which is usually called
    %\leanline{volume}). For the measure, we have: \leanline{structure MeasureTheory.Measure α [MeasurableSpace α] extends MeasureTheory.OuterMeasure}.
    %Here $\sigma$-additivity of a
    %\leanline{Measure α} reads
%\begin{minted}[mathescape, numbersep=5pt, framesep=5mm, bgcolor=mygray]{Lean}
%  m_iUnion : ∀ ⦃f : ℕ → Set α⦄,
%    (∀ (i : ℕ), MeasurableSet (f i)) →
%    Pairwise (Disjoint on f) →
%    ↑toOuterMeasure (⋃ (i : ℕ), f i) =
%    ∑' (i : ℕ), ↑toOuterMeasure (f i)
%\end{minted}    
  \end{enumerate}
\end{remark}

\noindent


\begin{remark}[Generated $\sigma$-algebra, image measure]
  We will frequently need two basic results: 
  \begin{enumerate}
  \item The intersection of any number of $\sigma$-algebras is again a
    $\sigma$-algebra. For this reason, if $\mathcal H \subseteq
    2^\alpha$, we define the $\sigma$-algebra $\sigma(\mathcal H) :=
    \bigcap_{\mathcal F \supseteq \mathcal H \text{ $\sigma$-algebra}}
    \mathcal F$. This is called the $\sigma$-algebra generated from
    $\mathcal H$; see \leanline{MeasurableSpace.generateFrom}.  \\ In
    particular, if $\mathcal O$ is a topology on $\alpha$, we call
    $\mathcal B := \sigma(\mathcal O)$ the Borel $\sigma$-algebra on
    $\alpha$. In \leanline{mathlib}, this is
\begin{minted}[mathescape, numbersep=5pt, framesep=5mm, bgcolor=mygray]{Lean}
def borel (α : Type u) [TopologicalSpace α] :
    MeasurableSpace α :=
  generateFrom { s : Set α | IsOpen s }
\end{minted}
    In addition, \leanline{mathlib} provides a similar notion, which is
\begin{minted}[mathescape, numbersep=5pt, framesep=5mm, bgcolor=mygray]{Lean}
class OpensMeasurableSpace (α : Type*)
    [TopologicalSpace α] [h : MeasurableSpace α] :
    Prop where borel_le : borel α ≤ h
\end{minted}
    Here, all open sets are measurable, so the $\sigma$-algebra defining
    \leanline{h} might be larger than the Borel $\sigma$-algebra. We will
    need both notions in our proofs.
  \item Let $(\alpha, \mathcal F)$ and $(\beta, \mathcal G)$ be
    measurable spaces. Some $f : \alpha \to \beta$ is called
    measurable (with respect to $\mathcal F$ and $\mathcal G$) if
    $f^{-1} \mathcal G \subseteq \mathcal F$. In \leanline{mathlib},
    see \leanline{Measurable}.
    %and note that this notion usually comes
    %with some \leanline{m : Measure α} and $f$ is almost everywhere
    %measurable (\leanline{AEMeasurable}) if it coincices with some
    %\leanline{Measurable g}, \leanline{m}-almost everywhere\todo{do we need aemeasurable? could it be omitted for simplicity?}.
    \\ In this
    case, if $\mu$ is a measure on $\mathcal F$, the measure $\nu:
    \mathcal G \to [0, \infty], \nu(B) := \mu (f^{-1}(B))$ is called
    the image (or push-forward) measure of $\mu$ under $f$. We write
    $\nu := f_\ast \mu$; see \leanline{MeasureTheory.Measure.map}.
  \end{enumerate}
\end{remark}

We now come to our contribution. 
The extension theorem is a statement about extending a
set-function on a product space to a (finite) measure, where the
product space can come with an arbitraty index set.  The next
definition covers the important concept of a projective family of
measures. In short, we define measures on any finite subset of indices
in a consistent way. Finite sets (of some type \leanline{ι}) are
formalized with \leanline{Finset ι}. % Such a \leanline{Finset} is
defined as a \leanline{Multiset} (a
%quotient type of the \leanline{List}-type with respect to permutations
%of the list) together with a proof that it contains no duplicates\todo{remove this implementation detail?}.

\begin{definition}[Projective family and projective limit] \mbox{}
  \begin{enumerate}
  \item For some set $\iota$, we will write $J\subseteq_f \iota$ if
    $J\subseteq \iota$ and $J$ is finite.
    %We will write $J\subseteq_c \iota$ if $J\subseteq \iota$ and $J$
    %is countable.
  \item Let $\iota$ be some (index) set and $(\alpha_i)_{i\in\iota}$ a
    family of sets. For $J\subseteq \iota$, we denote $\alpha_J :=
    \prod_{j \in J} \alpha_j$ and $\pi_J : \alpha_\iota \to \alpha_J$
    the projection. For $H\subseteq J \subseteq \iota$, we write
    $\pi_H^J$ for the projection $\alpha_J \to \alpha_H$.
  \item Let $\mathcal F_i$ be a $\sigma$-algebra in $\alpha_i$,
    $i\in\iota$. For $J\subseteq_f \iota$, let $\mathcal F_J$ be the
    product-$\sigma$-algebra on $\alpha_J$, and $\mathcal F_\iota$ be
    the $\sigma$-algebra generated by cylinder sets
    $\{\pi_J^{-1}\prod_{j\in J} A_j: J \subseteq_f \iota, A_j \in
    \sigma(E_j), j\in J\}$.
  \item A family $(P_J)_{J\subseteq_f I}$, where $P_J$ is a finite
    measure on $\mathcal F_J$, is called projective if
    $$P_H = (\pi_H^{J})_\ast P_J$$ for all
    $H\subseteq J \subseteq_f I$. (Recall that $A \mapsto
    (\pi_H^{J})_\ast P_J(A) := P_J((\pi_H^{J})^{-1}A)$
    is called the image measure of $P_J$ under $\pi_H^{J}$.)
  \item If, for some projective family $(P_J)_{J\subseteq_f \iota}$,
    there is a finite measure $P_\iota$ on $\mathcal F_\iota$ with
    $P_J = (\pi_J)_\ast P_\iota$ for all $J\subseteq_f \iota$, then we
    call $P_\iota$ projective limit of $(P_J)_{J\subseteq_f \iota}$.
  \end{enumerate}
\end{definition}

In contrast to statements already part of
\leanline{mathlib}, in the sequel we \leanlinecolor{highlight}
definitions and statements which are part of our own code. In the
formalization of the above, we use \leanline{variable {ι : Type _} {α: ι → Type _}},
which fixes the index set $\iota$ and all spaces
$\alpha_t, t\in\iota$ as global variables.
In Lean, the product space is then \leanline{∀ j : ι, α j} and the product over a finite set \leanline{J} is \leanline{∀ j : J, α j}.

We formalize the projective property as follows. This definition is not specific to measures but works as long as we have a preorder
(which is the subset relation on \leanline{Finset ι} below).

\begin{minted}[highlightlines={1}, mathescape, numbersep=5pt, framesep=5mm, bgcolor=mygray]{Lean}
def IsProjective [Preorder ι]
    (P : ∀ j : ι, α j) (π : ∀ {i j : ι}, j ≤ i → α i → α j) : Prop :=
    ∀ (i j) (hji : j ≤ i), P j = π hji (P i)
\end{minted}

\noindent
With this, we can define the projective family as follows. The typeclass inference system of
\leanline{Lean} automatically uses the subset relation to generate
\leanline{[Preorder (Finset ι)]} when \leanline{isProjective} is
called.

\begin{minted}[highlightlines={1}, mathescape, numbersep=5pt, framesep=5mm, bgcolor=mygray]{Lean}
def IsProjectiveMeasureFamily
    [∀ i, MeasurableSpace (α i)]
    (P : ∀ J : Finset ι, Measure (∀ j : J, α j)) :
    Prop :=
  IsProjective P (fun I _ hJI μ => μ.map
    fun x : ∀ i : I, α i => fun j => x ⟨j, hJI j.2⟩ :
    ∀ (I J : Finset ι) (_ : J ⊆ I), Measure (∀ i : I, α i)
    → Measure (∀ j : J, α j))
\end{minted}

\noindent
It is worth understanding the precise connection of
\leanlinecolor{isProjective} and
\leanlinecolor{isProjectiveMeasureFamily}. In the latter, the first
variable of \leanlinecolor{isProjective} is the family \leanline{P} of
finite measures for all finite subsets of \leanline{ι}. The second
variable is the functions which maps two sets \leanline{I J : Finset
  ι} and a proof \leanline{hJI} of \leanline{J ⊆ I} together with
\leanline{P I} (which is \leanline{μ} in the statement) to the image
measure on \leanline{J}, which is \leanline{μ.map (fun x : ∀ i : I, α
  i => fun j => x ⟨j, hJI j.2⟩)}.  The map defined by the
\leanline{=>}-notation maps every $(x_i)_{i\in I}$ to a function of
\leanline{j}, whose type is the subtype of \leanline{I}, consisting of
a value and a proof of \leanline{J ⊆ I}. In other words, this is
$(x_j)_{j\in J}$.

\noindent
Now we are ready to formulate the Kolmogorov extension theorem:

\begin{theorem}[Kolmogorov extension]\label{T1}
  For all $t\in\iota$, let $\alpha_t$ be a separable, complete
  pseudo-extended-metric space and $\mathcal F_t$ the Borel
  $\sigma$-algebra generated by its topology. Let $(P_J)_{J\subseteq_f
    \iota}$ be a projective family of finite measures and $P$ be
  defined on $\mathcal A := \bigcup_{J \subseteq_f \iota} \mathcal
  F_J$ given by $P(A) = P_J(\pi_J A)$ for $A\in\mathcal F_J$. Then,
  there is a unique extension of $P$ to $\sigma(\mathcal A)$.
\end{theorem}

Rather than giving the formalization of this theorem, we give the
definition of the resulting measure (which is the projective
limit). We give the formalized proof at the end of this section, since
we first have to provide a formalization of all tools needed in the
proof.

\begin{minted}[highlightlines={1}, mathescape, numbersep=5pt, framesep=5mm, bgcolor=mygray]{Lean}
def projectiveLimitWithWeakestHypotheses
    [∀ i, PseudoEMetricSpace (α i)]
    [∀ i, BorelSpace (α i)]
    [∀ i, SecondCountableTopology (α i)]
    [∀ i, CompleteSpace (α i)] [∀ i, Nonempty (α i)]
    (P : ∀ J : Finset ι, Measure (∀ j : J, α j))
    [∀ i, IsFiniteMeasure (P i)]
    (hP : IsProjectiveMeasureFamily P) :
    Measure (∀ i, α i)
\end{minted}

We note that we extend the standard assumption that all $\alpha_t$ are
separable, complete metric spaces (or Polish, i.e.\ separable and
metrizable through a complete metric) to cover the case of extended
pseude-metric spaces. Such spaces do not satisfy the frequently used
Hausdorff (or t2) property, i.e.\ there can be $x\neq y$ such that all
open balls around $x$ and $y$ overlap. {\color{blue}We note that this
  generalization was only possible since underlying results in
  \leanline{mathlib} were already provided on the same level of
  generality: More precisely,
  \leanline{isCompact_iff_totallyBounded_isComplete}, which shows that
  a set $A \subseteq \alpha$ is compact iff it is complete and totally
  bounded {\color{blue}(see note~\ref{note:tot})}, requires $\alpha$
  to be a uniform space (recall that every metric space is uniform);
  recall such spaces from Remark~\ref{rem:uniform}, and see
  note~\ref{note:tot} for a definition of total boundedness. Since we
  also require the underlying space(s) to be second countable (see
  also in the proof of Lemma~\ref{L:relcoPol}), we have to make a
  countabiliy assumption, which leads us to (extended) pseudo-metric
  spaces by \leanline{UniformSpace.pseudoMetrizableSpace}; see also
  Remark~\ref{rem:uniform} for some more details.}

\subsection{Extending a set function}
In the formulation of Theorem~\ref{T1}, we extend $P$, which is
defined on a union of $\sigma$-algebras. However, unions of
$\sigma$-algebras in general are not $\sigma$-algebras, but they can
be used to define the $\sigma$-algebra generated by the union. So, we
need to extend $P$ to the $\sigma$-algebra generated by $\mathcal
A$. This is exactly what Carathéodory's extension theorem is made
for. In fact, we implemented this result in greater generality than
needed for the proof of the extension theorem. Note that $\mathcal A$
in Theorem~\ref{T1} is a ring of sets (see the next definition)
containing the whole set. (This is sometimes called a field of sets.)
We will work with the weaker semi-ring as introduced next in the
formulation of Carathéodory's extension theorem; see
e.g.\ \cite[Definition~1.9]{Klenke2013}.

\begin{definition}[Semi-ring, ring]\label{def:semi}
  Let $\alpha$ be some set. We call $\mathcal H \subseteq 2^\alpha$ a
  \emph{semi-ring}, if it is (i) a $\pi$-system (i.e.\ closed under
  $\cap$) and (ii) for all $A, B \in\mathcal H$ there is $\mathcal K
  \subseteq_f \mathcal H$ with\footnote{We write $A\uplus B$ for
    $A\cup B$ if $A\cap B=\emptyset$.}  $B\setminus A = \biguplus_{K
    \in \mathcal K} K$.  \\ We call $\mathcal H \subseteq 2^\alpha$ a
  \emph{ring}, if it is closed under $\cup$ and under set-differences.
\end{definition}

\noindent
Any ring is a semi-ring since $A\cap B = A \setminus (A \setminus B)$,
i.e.\ every ring is a $\pi$-system.
Using \leanline{variable {α : Type _} {s t : Set α}}, we define a semi-ring as follows:

\begin{minted}[highlightlines={1}, mathescape, numbersep=5pt, framesep=5mm, bgcolor=mygray]{Lean}
structure SetSemiring (C : Set (Set α)) : Prop where
  empty_mem : ∅ ∈ C
  inter_mem : ∀ (s) (_ : s ∈ C) (t) (_ : t ∈ C), s ∩ t ∈ C
  diff_eq_Union' :
    ∀ (s) (_ : s ∈ C) (t) (_ : t ∈ C),
    ∃ (I : Finset (Set α)) (_h_ss : ↑I ⊆ C)
    (_h_dis : PairwiseDisjoint (I : Set (Set α)) id),
      t \ s = ⋃₀ I
\end{minted}

Let us remark that (i) \leanline{Lean} indicates a coercion by
\leanline{↑}, which in this case is from \leanline{Finset} to
\leanline{Set} and (ii) we have \leanline{PairwiseDisjoint (s : Set ι) (f : ι → α)}
iff the images of any distinct elements of \leanline{ι}
under \leanline{f} are different. (Hence, if \leanline{f = id}, the
usual definition of pairwise disjoint sets unfolds.)

\noindent
We do not show the formalization of rings here.
The important ring of sets in our formalization is the ring of measurable cylinders on a product space.
A cylinder is defined as follows:

\begin{minted}[highlightlines={1}, mathescape, numbersep=5pt, framesep=5mm, bgcolor=mygray]{Lean}
def cylinder
    (s : Finset ι) (S : Set (∀ i : s, α i)) :
    Set (∀ i : ι, α i) :=
  (fun (f : ∀ j : ι, α j) => fun i : s => f i) ⁻¹' S
\end{minted}

\noindent
In this definition, \leanline{s} is a finite subset of the index set
\leanline{ι}, and for some $S$ in the finite product $\prod_{i \in s}
\alpha_i$, consider the projection $\pi_s : \prod_{i \in \iota}
\alpha_i \to \prod_{i \in s} \alpha_i$, and consider the preimage of
$S$ (This is what the last line in previous definition gives.).
We can define \leanlinecolor{cylinders α}, the set of all measurable cylinders.

\begin{minted}[highlightlines={1}, mathescape, numbersep=5pt, framesep=5mm, bgcolor=mygray]{Lean}
def cylinders : Set (Set (∀ i : ι, α i)) :=
  ⋃ (s) (S) (_ : MeasurableSet S), {cylinder s S}
\end{minted}

Membership of that set is characterized as follows:
\begin{minted}[highlightlines={1}, framesep = 0pt, mathescape, numbersep=5pt, framesep=5mm, bgcolor=mygray]{Lean}
theorem mem_cylinders
    (t : Set (∀ i : ι, α i)) :
    t ∈ cylinders α ↔
      ∃ (s S : _) (_ : MeasurableSet S), t = cylinder s S
\end{minted}
\noindent
The finite set \leanline{s} of the potential coordinates where
the cylinder deviates from the whole set needs not be unique.
From the exists-statement of \leanline{mem_cylinders}, we however choose an
arbitrary such \leanline{Finset ι}, called \leanlinecolor{cylinders.finset}.


The set-system of cylinders is in fact a field, hence a ring and
a semi-ring. This means we can prove the following:

\begin{minted}[highlightlines={1-3}, mathescape, numbersep=5pt, framesep=5mm, bgcolor=mygray]{Lean}
theorem setField_cylinders : SetField (cylinders α) 
theorem setRing_cylinders :  SetRing (cylinders α)
theorem setSemiringCylinders : SetSemiring (cylinders α)   
\end{minted}

From the field/ring/semi-ring of cylinders, we have to define the
generated $\sigma$-algebra. This uses the following:
\begin{minted}[highlightlines={1}, framesep = 0pt, mathescape, numbersep=5pt, framesep=5mm, bgcolor=mygray]{Lean}
theorem generateFrom_cylinders :
    MeasurableSpace.generateFrom (cylinders α)
    = MeasurableSpace.pi 

instance MeasurableSpace.pi
    [m : ∀ a, MeasurableSpace (π a)] :
    MeasurableSpace (∀ a, π a) :=
  iSup a, (m a).comap fun b => b a
\end{minted}
Here, \leanline{comap {β : Type} (f : α → β) (m : MeasurableSpace β)}
is the measurable space consisting of pre-images of measurable subsets
of \leanline{β}, and the set of $\sigma$-algebras on the product
$\prod_{i\in\iota} \alpha_i$ (or on any other space) is a complete
lattice, i.e.\ the subset defined by the \leanline{comap}s has a
supremum, which defines the $\sigma$-algebra generated by the
cylinders.

Let us state an important lemma on semi-rings: 

\begin{lemma}\label{l1}
  Let $\mathcal H$ be a semi-ring, $\mathcal I \subseteq_f \mathcal H$,
  $A \in \mathcal H$. Then, there is $\mathcal K \subseteq_f \mathcal
  H$ such that $\mathcal K$ contains pairwise disjoint sets and $A
  \setminus \bigcup_{I \in \mathcal I} I = \biguplus_{K\in \mathcal K}
  K$.
\end{lemma}

\begin{proof}
  We proceed by induction on $\mathcal I$. If $\mathcal I$ is a
  singleton, the assertion is true by the definition of a semi-ring. If
  it holds for some $\mathcal I$ (i.e.\ there is $\mathcal K
  \subseteq_f \mathcal H$ with $A \setminus \bigcup_{I \in \mathcal I}
  I = \biguplus_{K\in \mathcal K} K$), let us consider $\mathcal I' =
  \{J\} \cup \mathcal I$ for some $J \notin \mathcal I$. For each $K
  \in \mathcal K$, Write $K \setminus J = \biguplus_{J_K \in \mathcal
    J_K} J_K$ for some $\mathcal J_K \subseteq_f \mathcal H$ (which
  exists by the definition of a semi-ring). Then, write
  \begin{align*}
    A \setminus \bigcup_{I' \in \mathcal I'} I & = \Big(A \setminus
    \bigcup_{I \in \mathcal I} I\Big) \setminus J = \biguplus_{K\in
      \mathcal K} K \setminus J = \biguplus_{K\in \mathcal K}
    \biguplus_{J_K \in \mathcal J_K} J_K.
  \end{align*}
  This concludes the proof, since the latter disjoint union is over a
  finite set.
\end{proof}

The proof as well as its formalization is somewhat straight-forward,
but requires induction over finite sets. 

\begin{minted}[highlightlines={1}, mathescape, numbersep=5pt, framesep=5mm, bgcolor=mygray]{Lean}
theorem exists_disjoint_finset_diff_eq
    (hC : SetSemiring C) (hs : s ∈ C)
    (I : Finset (Set α)) (hI : ↑I ⊆ C) :
    ∃ (J : Finset (Set α)) (_h_ss : ↑J ⊆ C)
    (_h_dis : PairwiseDisjoint (J : Set (Set α)) id),
    s \ ⋃₀ I = ⋃₀ J
\end{minted}

\noindent
The existence-statement of the above lemma actually gives rise to a
definition. Here, we use \leanline{Exists.choose} in order to extract
an element from an $\exists$-statement. In addition, we do not allow
$\emptyset \in \mathcal K$ without loss of generality:

\begin{minted}[highlightlines={1}, mathescape, numbersep=5pt, framesep=5mm, bgcolor=mygray]{Lean}
def diff₀ (hC : SetSemiring C)
    (hs : s ∈ C) (I : Finset (Set α)) (hI : ↑I ⊆ C)
    [DecidableEq (Set α)] : Finset (Set α) :=
  (hC.exists_disjoint_finset_diff_eq hs I hI).choose \ {∅}
\end{minted}    

Extending Lemma~\ref{l1}, we would like to write a finite union of
members of a semi-ring as a finite union of disjoint sets\todo{Is this section going into too much detail?}.


\begin{lemma}\label{l2}
  Let $\mathcal H$ be a semi-ring and $A_1,...,A_m \in \mathcal
  H$. Then, there are $\mathcal K_1,...,\mathcal K_m \subseteq_f
  \mathcal H$ disjoint such that $\bigcup_{n=1}^m \mathcal K_n$
  contains disjoint sets and $\bigcup_{m=1}^n A_m = \biguplus_{m=1}^n
  \biguplus_{K \in \mathcal K_n} K$.  
\end{lemma}

\begin{proof}
  Indeed, we may write $ \bigcup_{m=1}^n A_m = \biguplus_{n=1}^m
  \Big(A_n \setminus \bigcup_{i=1}^{n-1}A_i\Big).$ Then, the result
  follows by applying Lemma~\ref{l1} to $A_n \setminus
  \bigcup_{i=1}^{n-1}A_i$, $n=1,...,m$.
\end{proof}

\noindent
The formalized version again gives rise to a definition:

\begin{minted}[highlightlines={1}, mathescape, numbersep=5pt, framesep=5mm, bgcolor=mygray]{Lean}
def indexedDiff₀ (hC : SetSemiring C)
    (J : Finset (Set α)) (hJ : ↑J ⊆ C)
    (n : Fin J.card) : Finset (Set α) :=
  hC.diff₀ (ordered_mem' hJ n) (finsetLT J n)
    (finsetLT_subset' J hJ n)
\end{minted}

Here, recall that \leanline{Fin n} is the subtype of \leanline{Nat}
consisting of all numbers \leanline{<n}. In addition,
\leanline{ordered_mem' (hJ : ↑J ⊆ C) (n : Fin (card J)) : ↑(ordered J)
  n ∈ C} gives a proof that the \leanline{n}th element of \leanline{J}
is in \leanline{C}, \leanline{finsetLT (J : Finset (Set α)) (n : Fin (card J)) : Finset (Set α)} is the set consisting of the
\leanline{n} sets in \leanline{J} numbered $0,...,n-1$, and
\leanlinecolor{finsetLT_subset'} gives a proof that
\leanline{↑(finsetLT J n) ⊆ C}.  So, in terms of Lemma~\ref{l2}, this
gives, for $n=1,...,m$, a construction for $\mathcal K_n \subseteq_f
\mathcal H$.

\noindent
\subsection{Carathéodory's extension theorem}
Measures are set functions defined on a $\sigma$-algebra (i.e.\ an
algebra stable under countable unions), satisfying some properties
which we recall below. Mostly, defining such a measure on the full
$\sigma$-algebra is not possible, but defining a set function on a
semi-ring is possible. As an example, consider Lebesgue-measure on
$\mathbb R$, with the Borel $\sigma$-algebra on $\mathbb R$, which is
$\sigma(\mathcal O)$, where $\mathcal O$ is the set of open sets, and
$\sigma(\mathcal O)$ is the smallest $\sigma$-algebra containing all
open sets. Since this set system is defined only abstractly, it is
hard to know which volume each of these sets should be assigned to at
first sight. However, the volume of an interval is easy, since it may
be defined by the length of the interval. So, in order to construct
measures from a set function $m$ on a (semi-)ring $\mathcal H$
(e.g.\ the set of all semi-open intervals), it has been a fundamental
insight of Carathéodory that one may start by defining a set function
(outer measure) $\mu$ on $2^\alpha$, and then show $\mu$ extends $m$
and defines a measure on $\sigma(\mathcal H)$. (Note that the set of
semi-open intervals generates the Borel $\sigma$-algebra on $\mathbb
R$.) We will follow this abstract construction, and start by stating
some basic concepts.

\begin{definition}\label{def:content}
  For some set $\alpha$, let $\mathcal H \subseteq 2^\alpha$
  and call any $m : \mathcal H \to [0,\infty]$ a
    content.
  \begin{enumerate}
  \item $m$ is called additive if for $\mathcal K \subseteq_f \mathcal
    H$ pariwise disjoint and $\bigcup_{K \in \mathcal K} K \in
    \mathcal H$, we have $m \Big(\bigcup_{K \in \mathcal K} K \Big) =
    \sum_{K \in \mathcal K} m(K)$. If the same holds for\footnote {We
      write $A \subseteq_c B$ if $A$ is a countable subset of
      $B$.}$\mathcal K \subseteq_c \mathcal H$ pariwise disjoint, we
    say that $m$ is $\sigma$-additive.
  \item The set-function $m$ is called sub-additive if for $\mathcal K
    \subseteq_f \mathcal H$ and $\bigcup_{K \in \mathcal K} K \in
    \mathcal H$, we have $m \Big(\bigcup_{K \in \mathcal K} K \Big)
    \leq \sum_{K \in \mathcal K} m(K)$. (Note that elements of
    $\mathcal K$ need not be disjoint.) Here, $\sigma$-sub-additivity
    is defined in the obvious way using $\mathcal K\subseteq_c
    \mathcal H$.
  \item If $m(A) \leq m(B)$ for $A\subseteq B$ and $A,B\in\mathcal H$,
    we say that $m$ is monotone.
  \item If $\mathcal H$ is a $\sigma$-algebra and $m$ is
    $\sigma$-additive with $m(\emptyset) = 0$, we call
    $m$ a measure.
  \item If $\mathcal H = 2^\alpha$ and $m$ is monotone and
    $\sigma$-sub-additive with $m(\emptyset)=0$, we call $m$ an outer
    measure.
  \end{enumerate}
\end{definition}

\noindent
For additive contents, we need two
definitions. \leanlinecolor{AddContent} defines a set-function
\leanline{toFun} on \leanline{Set α}, whereas
\leanlinecolor{extendContent} comes with \leanline{m : ∀ s : Set α, s ∈ C → ℝ≥0∞}.

\begin{minted}[highlightlines={1, 10}, mathescape, numbersep=5pt, framesep=5mm, bgcolor=mygray]{Lean}
structure AddContent (C : Set (Set α)) where
  toFun : Set α → ℝ≥0∞
  empty' : toFun ∅ = 0
  add' : ∀ (I : Finset (Set α))
    (_h_ss : ↑I ⊆ C) (_h_dis : PairwiseDisjoint
      (I : Set (Set α)) id)
    (_h_mem : ⋃₀ ↑I ∈ C),
    toFun (⋃₀ I) = ∑ u in I, toFun u

def extendContent
    (hC : SetSemiring C)
    (m : ∀ s : Set α, s ∈ C → ℝ≥0∞)
    (m_empty : m ∅ hC.empty_mem = 0)
    (m_add :
      ∀ (I : Finset (Set α)) (h_ss : ↑I ⊆ C)
      (_h_dis : PairwiseDisjoint (I : Set (Set α)) id)
      (h_mem : ⋃₀ ↑I ∈ C),
      m (⋃₀ I) h_mem = ∑ u : I, m u (h_ss u.prop)) :
    AddContent C 
\end{minted}

For the concrete application we have in mind, we introduce a
definition which uses a specific semi-ring, the measurable cylinders,
based on a projective family of measures.  That is the goal of the two
definitions \leanlinecolor{kolmogorovFun} and
\leanlinecolor{kolContent} below:

\begin{minted}[highlightlines={1,7}, mathescape, numbersep=5pt, framesep=5mm, bgcolor=mygray]{Lean}
def kolmogorovFun
    (P : ∀ J : Finset ι, Measure (∀ j : J, α j))
    (s : Set (∀ i, α i))
    (hs : s ∈ cylinders α) : ℝ≥0∞ :=
  P (cylinders.finset hs) (cylinders.set hs)

def kolContent
    (hP : IsProjectiveMeasureFamily P) :
    AddContent (cylinders α) :=
  extendContent setSemiringCylinders
    (kolmogorovFun P) (kolmogorovFun_empty hP)
    (kolmogorovFun_additive hP)
\end{minted}

Here, \leanlinecolor{kolmogorovFun_empty} and
\leanlinecolor{kolmogorovFun_additive} give proofs for
\leanline{m_empty} and \leanline{m_add} in
\leanlinecolor{extendContent} as applied to a
\leanlinecolor{kolmogorovFun}.

\noindent
From the next lemma, we will need monotonicity and sub-additivity, as
well as $\sigma$-additive $\Rightarrow$ $\sigma$-sub-additive in the
proof of the Carathéodory extension,
Theorem~\ref{T:masseind}. We do not show any details
  about its formalization here.

\begin{lemma}[Set-functions on semi-rings]\label{l:halbextRing}
  Let $\mathcal H$ be a semi-ring and $m: \mathcal H\to [0,\infty]$
  additive. Then, $m$ is monotone and sub-additive. In addition, $m$
  is $\sigma$-additive iff it is $\sigma$-sub-additive.
\end{lemma}

\begin{proof}
  We start by monotonicity. Let $A, B\in\mathcal H$ with $A\subseteq
  B$ and $\mathcal K \subseteq_f \mathcal H$ with $B \setminus A =
  \biguplus_{K\in\mathcal K} K$. Therefore, we can write $m(A) \leq
  m(A) + \sum_{K\in\mathcal K} m(K) = m(B)$.

  Next, we claim that for $\biguplus_{I\in\mathcal I} I \subseteq A$
  with all sets belonging to $\mathcal H$, we have $\sum_{I \in \mathcal
    I} m(I) \leq m(A)$. For this, write $A \setminus \biguplus_{I\in
    \mathcal I} I = \biguplus_{K\in\mathcal K} K$ as in
  Lemma~\ref{l1}. Then,
  $$ m(A) = m\Big(\biguplus_{I\in\mathcal I} I \uplus
  \biguplus_{K\in\mathcal K} K\Big) = \sum_{I\in\mathcal I} m(I) +
  \sum_{K\in\mathcal K} m(K) \geq \sum_{I\in\mathcal I} m(I). $$
  
  For sub-additivity, let $\mathcal I \subseteq_f \mathcal H$ with
  $\bigcup_{I \in \mathcal I} I \in \mathcal H$. Without loss of
  generality, we write $\mathcal I = \{I_1,...,I_n\}$ for some $n$. We
  need to show $m\Big(\bigcup_{k=1}^n I_i\Big) \leq \sum_{k=1}^n
  m(I_k)$. For $k=2,...,n$, we write
  $$ \bigcup_{k=1}^n I_k = \biguplus_{k=1}^n \Big(I_k \setminus
  \bigcup_{j=1}^{k-1} I_j\Big) = \biguplus_{k=1}^n \biguplus_{K_k \in
    \mathcal K_k} K_k$$ with $\mathcal K_k$ as in Lemma~\ref{l1}. So,
  since $\biguplus_{K_k \in \mathcal K_k} K_k \subseteq I_k \in
  \mathcal H$,
  $$ m\Big(\bigcup_{k=1}^n I_i\Big) = \sum_{k=1}^n \sum_{K_k \in
    \mathcal K_k} m(K_k) \leq \sum_{k=1}^n m(I_k).$$

  \noindent
  Now, we show that $m$ is $\sigma$-additive $\iff$ it is
  $\sigma$-sub-additive.

  \noindent
  '$\Rightarrow$': Here, just copy the proof of sub-additivity, but
  using countable $\mathcal I$, i.e.\ $n=\infty$. For '$\Leftarrow$',
  let $\mathcal I \subseteq_c \mathcal H$ and consist of
  disjoint sets with $A = \biguplus_{I \in \mathcal I} I \in\mathcal
  H$. Since $m$ is monotone and for any $\mathcal I' \subseteq_f
  \mathcal I$, we have $\biguplus_{I\in\mathcal I'} I \subseteq A$
  (hence $\sum_{I \in \mathcal I'} m(I') \leq m(A)$),
  $$ \sum_{I \in \mathcal I} m(I) = \sup_{\mathcal I' \subseteq_f
    \mathcal I} \sum_{I \in \mathcal I'} m(I) \leq m(A) \leq \sum_{I
    \in \mathcal I} m(I)$$ by $\sigma$-sub-additivity. So,
  $\sigma$-additivity follows.
\end{proof}

\noindent
Although some material on outer measures was covered in {\tt mathlib}
already, the classical extension theorem (extending a set function $m$
on a semi-ring $\mathcal H$)) was not provided yet. In particular, this
result states that the outer measure coincides with $m$ on $\mathcal
H$. All statements on equality of $\mu$ and $m$ present in {\tt
  mathlib} at the time of writing have too many requirements: they all require $m$ to be a $\sigma$-additive function defined on a $\sigma$-algebra.

The next result extends a set function on a semi-ring to an outer measure.

\begin{proposition}[Outer measure induced by a set function on a semi-ring] %\mbox{}
  Let \label{P:auss} $\mathcal H$ be a semi-ring and $m: \mathcal
  H\to\mathbb R_+$ additive. For $A\subseteq E$ let
  $$ \mu(A) := \inf_{\mathcal G \in \mathcal U(A)}
    \sum_{G\in\mathcal G} m(G)$$ where
  $$ \mathcal U(A) := \big\{\mathcal G \subseteq_c \mathcal H,
    A\subseteq \bigcup_{G\in\mathcal G} G\big\}$$ is the set of
    countable coverings of $A$. Then, $\mu$ is an outer measure.
\end{proposition}

\sloppy Let us now formulate the classical results by
Carathéodory. The first leads to the definition of the measurable
space \leanline{OuterMeasure.caratheodory}, covered in
\leanline{mathlib}. See e.g.\ \cite[Theorem 2.1]{Kallenberg2020}

\begin{theorem}[\boldmath $\mu$-measurable sets are a
    $\sigma$-algebra]\label{T:cara} Let $\mu$ be an outer measure on
  $E$ and $\mathcal F$ the set of $\mu$-measurable sets,
  i.e.\ the set of sets $A$ satisfying
  \begin{align}
    \label{eq:mumess}
    \mu(B) = \mu(B\cap A) + \mu(B\cap A^c), \qquad B \subseteq E.
  \end{align}
  Then, $\mathcal F$ is a $\sigma$-Algebra and $\mu|_{\mathcal F}$ is
  a measure. In addition, $\{N\subseteq \Omega:
  \mu(N)=0\}\subseteq \mathcal F$, i.e.\ $\mathcal F$ is complete.
\end{theorem}

\noindent
\sloppy The second result states that for an outer measure induced by
an additive content on a semi-ring, we have $\sigma(\mathcal H)
\subseteq \mathcal F$. In particular, we then have defined a measure
on $\sigma(\mathcal H)$. This result is not yet covered in {\tt
  mathlib}. However, for product spaces, in
\leanline{MeasureTheory.Constructions.Pi}, there is
\leanline{pi_caratheodory}, which shows that $\sigma(\mathcal H)
\subseteq \mathcal F$ in {\color{blue} the construction of a product
  measure for a finite index set}. In addition, there is
\leanline{pi_pi_aux}, which shows that $\mu$ extends $m$ on $\mathcal
H$ in the same special case. See also \cite[Theorem
  2.5]{Kallenberg2020}.

\begin{theorem}[Carathéodory extension]\label{T:masseind}
  Let $\mathcal H$ be a semi-ring and $m: \mathcal H\to\mathbb R_+$
  $\sigma$-finite and $\sigma$-additive. Furthermore, let $\mu$ be the
  induced outer measure from Proposition~\ref{P:auss} and $\mathcal F$
  the $\sigma$-algebra from Theorem~\ref{T:cara}. Then,
  $\sigma(\mathcal H)\subseteq\mathcal F$ and $\mu$ coincides with $m$
  on $\mathcal H$.
\end{theorem}

\begin{proof}
  First, $m$ is $\sigma$-sub-additive by
  Lemma~\ref{l:halbextRing}.

  ~

  \noindent\emph{Step 1: $\mu|_{\mathcal H} = m$:} Let $A\in\mathcal
  H$. Choose $\mathcal K \subseteq_c \mathcal H$ with
  $A\subseteq \bigcup_{K\in\mathcal K} K$ and
  $$\mu(A) \geq \sum_{K\in\mathcal K} m(K) - \varepsilon.$$ By $ A =
  \bigcup_{K\in\mathcal K} K\cap A$,
  \begin{align*}
    \mu(A) \leq m(A) \leq \sum_{K\in\mathcal K} m(K\cap A) \leq
    \sum_{K\in\mathcal K} m(K) \leq \mu(A) + \varepsilon,
  \end{align*}
  where we have used $\sigma$-sub-additivity of $m$ in the second and
  monotonicity of $m$ in the third $\leq$ (see
  Lemma~\ref{l:halbextRing}).  With $\varepsilon\to 0$ we find that
  $\mu(A) = m(A)$.

  ~

  \noindent\emph{Step 2: $\sigma(\mathcal H)\subseteq \mathcal F$:}
  Let $B \subseteq E$, $A\in\mathcal H$ and $\varepsilon>0$. Choose
  $\mathcal K \subseteq_c \mathcal H$ such that $B \subseteq
  \bigcup_{K \in \mathcal K} K$ and $\mu(B) \geq \sum_{K\in\mathcal K}
  m(K) - \varepsilon.$ Then, by additivity of $m$,
  \begin{align*}
    \mu(B)+\varepsilon & \geq \sum_{K\in\mathcal K} \mu(K) =
    \sum_{K\in\mathcal K} \mu(K\cap A) + \sum_{K\in\mathcal K}
    \mu(K\cap A^c) \\ & \geq \mu(B\cap A) + \mu(B\cap A^c).
  \end{align*}
  By sub-additivity of $\mu$, we find that $\mu(B) \leq \mu(B\cap A) +
  \mu(B \cap A)$, so letting $\varepsilon\to 0$ leads to $\mu(A) =
  \mu(E\cap A) + \mu(E\cap A^c)$. This implies that $A$ is $\mathcal
  F$-measurable, and we have shown $\sigma(\mathcal H) \subseteq
  \mathcal F$, since $\mathcal F$ is a $\sigma$-algebra.
\end{proof}

Here is a formalization of the measure resulting from
Theorem~\ref{T:masseind}:

%% \begin{minted}[highlightlines={1}, mathescape, numbersep=5pt, framesep=5mm, bgcolor=mygray]{Lean}
%% def ofAddSubaddCaratheodory
%%     (hC : SetSemiring C)
%%     (m : ∀ s : Set α, s ∈ C → ℝ≥0∞)
%%     (m_empty : m ∅ hC.empty_mem = 0)
%%     (m_add : ∀ (I : Finset (Set α)) (h_ss : ↑I ⊆ C)
%%       (_h_dis : PairwiseDisjoint (I : Set (Set α)) id)
%%       (h_mem : ⋃₀ ↑I ∈ C),
%%       m (⋃₀ I) h_mem = ∑ u : I, m u (h_ss u.prop))
%%     (m_sigma_subadd : ∀ ⦃f : ℕ → Set α⦄
%%       (hf : ∀ i, f i ∈ C) (hf_Union : (⋃ i, f i) ∈ C),
%%       m (⋃ i, f i) hf_Union ≤ ∑' i, m (f i) (hf i)) :
%%     @Measure α (inducedOuterMeasure m
%%       hC.empty_mem m_empty).caratheodory
%% \end{minted}


%% \noindent
%% Here, \leanline{MeasureTheory.OuterMeasure.caratheodory}\todo{used in
%%   dot notation... but we don't want to explain this} is already
%% implemented and gives the assertion that $\mathcal F$ from
%% Theorem~\ref{T:cara} is a $\sigma$-algebra. Note that
%% \leanline{@Measure} is the same as \leanline{Measure}, but all
%% variables have to be given explicitely. The second argument is a proof
%% that measurable sets form a $\sigma$-algebra which cannot be inferred
%% by the type class system of \leanline{Lean}.

%% \noindent
%% In the proof of Kolmogorov's extension theorem, we use a closely
%% related definition\todo{should we present only this one and omit the lemma above?}:

\begin{minted}[highlightlines={1}, mathescape, numbersep=5pt, framesep=5mm, bgcolor=mygray]{Lean}
def Measure.ofAddContent
    [mα : MeasurableSpace α] (hC : SetSemiring C)
    (hC_gen : MeasurableSpace.generateFrom C = mα)
    (m : AddContent C)
    (m_sigma_subadd : ∀ ⦃f : ℕ → Set α⦄
      (hf : ∀ i, f i ∈ C) (hf_Union : (⋃ i, f i) ∈ C),
      m (⋃ i, f i) hf_Union ≤ ∑' i, m (f i) (hf i)) :
    Measure α
\end{minted}

\subsection{$\sigma$-additivity of set functions}
For the proof of Kolmogorov's extension Theorem, note that $P$ as
given in Theorem~\ref{T1} is a finite additive set function on the
ring $\mathcal A$. In order to use Theorem~\ref{T:masseind}, we
therefore have to show $\sigma$-additivity. For this, we will use
Lemma~\ref{l:stetigcompact} below, i.e.\ inner regularity of $P$ with
respect to compact sets. Before, we will show useful alternative
conditions for $\sigma$-additivity, which do not make use of any
topological structure of the underlying space.

\begin{lemma}[$\sigma$-additivity and continuity at $\emptyset$]\label{P:stetigmass}
  Let $\mathcal R$ be a ring and $m:\mathcal R\to \mathbb R_+$
  additive. Then, the following are equivalent:
  \begin{enumerate}
    \item $m$ is $\sigma$-additive;
    \item $m$ is $\sigma$-sub-additive;
    \item $m$ is continuous from below, i.e.\ for $A, A_1, A_2,... \in
      \mathcal R$ with $A_1 \subseteq A_2 \subseteq \cdots$ and $A =
      \bigcup_{n=1}^\infty A_n$, we have $m(A) = \lim_{n\to\infty}
      m(A_n)$.
    \item $m$ is continuous from above in $\emptyset$, i.e.\ for $A_1,
      A_2,...\in\mathcal R$ with $A_1 \supseteq A_2 \supseteq ...$ and
      $\bigcap_{n=1}^\infty A_n = \emptyset$, we have
      $\lim_{n\to\infty} m(A_n)=0$.
  \end{enumerate}
\end{lemma}

\begin{proof}
  1.$\Leftrightarrow$2.\ was already shown in
  Lemma~\ref{l:halbextRing}, since $\mathcal R$ is a semi-ring.

  ~

  \noindent 1.$\Rightarrow$3.: For $A_1, A_2,...$ as in 3., we write
  (with $A_0 := \emptyset$)
  \begin{align*}
     m(A) & % = m\Big( \biguplus_{n=1}^\infty \Big(A_n \setminus
     %\Big(\bigcup_{l=1}^{k-1} A_l\Big)\Big)\Big)
     = \lim_{n\to\infty}
     m\Big( \biguplus_{k=1}^n A_k \setminus \Big(\Big(\bigcup_{l=1}^{k-1}
     A_l\Big)\Big)\Big) = \lim_{n\to\infty} m(A_n).
  \end{align*}
  
  \noindent 3.$\Rightarrow$1.: Let $A_1, A_2,... \in \mathcal R$ be
  disjoint and $A = \biguplus_{n=1}^\infty A_n$. Set $B_n :=
  \bigcup_{k=1}^n A_k$, such that $B_1, B_2,...$ satisfy 3. Hence,
  \begin{align*}
    m\Big( \biguplus_{n=1}^\infty A_n\Big) & = m\Big(
    \bigcup_{n=1}^\infty B_n\Big) = \lim_{n\to\infty} m\Big(
    \bigcup_{k=1}^n B_n\Big) \\ & = \lim_{n\to\infty} m\Big(
    \biguplus_{k=1}^n A_n\Big) = \sum_{n=1}^\infty m(A_n).
  \end{align*}

  \noindent 3.$\Rightarrow$4.: Let $A_1, A_2,\dots \in\mathcal R$ as
  in 4. Set $B_n := A_1\setminus A_n$. Then, $B=A_1,B_1,B_2,\dots
  \in\mathcal R$ satisfy 3., and therefore
  $$m(A_1) = \lim_{n\to\infty} m(B_n) = m(A_1) -
  \lim_{n\to\infty}m(A_n),$$ and 4.\ follows.

  ~

  \noindent 4.$\Rightarrow$3.: Let $A,A_1,A_2,\dots \in\mathcal R$ as
  in 3. Let $B_n := A\setminus A_n \in \mathcal R, n\in\mathbb
  N$. Then, $\bigcap_{n=1}^\infty B_n = \emptyset$, so $$0 =
  \lim_{n\to\infty} \mu(B_n) = \mu(A) - \lim_{n\to\infty} \mu(A_n),$$
  and 3.\ follows.
\end{proof}

\noindent
As an example, we give the formalization of $4.\Rightarrow 1.$:

\begin{minted}[highlightlines={1}, mathescape, numbersep=5pt, framesep=5mm, bgcolor=mygray]{Lean}
theorem sigma_additive_addContent_of_tendsto_zero
    (hC : SetRing C)
    (m : AddContent C)
    (hm_ne_top : ∀ {s} (_ : s ∈ C), m s ≠ ∞)
    (hm : ∀ ⦃s : ℕ → Set α⦄ (_ : ∀ n, s n ∈ C),
      Antitone s → (⋂ n, s n) = ∅ →
      Tendsto (fun n => m (s n)) atTop (nhds 0))
      ⦃f : ℕ → Set α⦄ (hf : ∀ i, f i ∈ C) (hUf : (⋃ i, f i) ∈ C)
      (h_disj : Pairwise (Disjoint on f)) :
    m (⋃ i, f i) = ∑' i, m (f i) 
\end{minted}

\noindent
Next, we will be extending our analysis to the case of a topological
space. We define inner (and outer) regularity of set functions.

\begin{definition}[Inner regularity] \label{def:innerreg}
  Let $\alpha$ be some set, equipped with a topology, and $m$ be a
  set-function on some $\mathcal H \subseteq 2^\alpha$.
  \begin{enumerate}
  \item Let $p, q : 2^\alpha \to \{\text{true, false}\}$. Then, $m$ is
    called inner regular with respect to $p$ and $q$, if 
    $$ m(A) = \sup\{m(F) : p(F) = \text{true}, F \subseteq A\}$$ for
    all $A \in \mathcal H$ with $q(A) = \text{true}$.
  \item If $q(A) = \text{true}$ iff $A$ is measurable, we neglect the
    {\em and $q$}. If $p(A) = \text{true}$ iff $A$ is closed (compact,
    closed and compact), we say that $m$ is inner regular with respect
    to closed (compact, compact and closed) sets.
  \end{enumerate}
\end{definition}

The above definition closely resembles its formalization in
\leanline{mathlib}:

\begin{minted}[highlightlines={1}, mathescape, numbersep=5pt, framesep=5mm, bgcolor=mygray]{Lean}
def MeasureTheory.Measure.InnerRegular
    {α : Type u_1}  {_ : MeasurableSpace α}
    (μ : Measure α) (p q : Set α → Prop) :=
  ∀ ⦃U⦄, q U → ∀ r < μ U, ∃ K, K ⊆ U ∧ p K ∧ r < μ K
\end{minted}

For the next result, recall that for compact sets $C_1, C_2,...$ with
$\bigcap_{n=1}^\infty C_n = \emptyset$, there is some $N$ with
$\bigcap_{n=1}^N C_n = \emptyset$. More generally, compact sets form a
compact system, which is defined as follows:

\begin{definition}
  Let $\mathcal C \subseteq 2^\alpha$. If, for all $C_1, C_2,...$ with
  $\bigcap_{n=1}^\infty C_n = \emptyset$, there is some $N$ with
  $\bigcap_{n=1}^N C_n = \emptyset$, we call $\mathcal C$ a {\em
    compact system}.
\end{definition}

Here is the formalization:

\begin{minted}[highlightlines={1}, mathescape, numbersep=5pt, framesep=5mm, bgcolor=mygray]{Lean}
def IsCompactSystem (p : Set α → Prop) : Prop :=
    ∀ C : ℕ → Set α, (∀ i, p (C i)) → ⋂ i, C i = ∅ →
    ∃ (s : Finset ℕ), ⋂ i ∈ s, C i = ∅
\end{minted}

\begin{lemma}\label{l:stetigcompact}
  Let $\alpha$ be a topological space and $\mu$ be an additive set
  function on a ring $\mathcal R$ contained by the Borel
  $\sigma$-algebra, and which is inner regular with respect to a
  compact system. Then, $\mu$ is $\sigma$-additive.
\end{lemma}

\begin{proof}
  According to Lemma~\ref{P:stetigmass}, we need to show continuity of
  $\mu$ in $\emptyset$, so let $A_1 \supseteq A_2 \supseteq... \in
  \mathcal R$ satisfy $\bigcap_{n=1}^\infty A_n = \emptyset$ and
  $\varepsilon>0$. Let $\delta_1, \delta_2,... >0$ with
  $\sum_{n=1}^\infty \delta_n < \varepsilon$. For each $n$, let
  $\mathcal C \ni C_n \subseteq A_n \in \mathcal R$ with $\mu(A_n)
  \leq \mu(C_n) + \delta_n$. We have that $\bigcap_{n=1}^\infty C_n
  \subseteq \bigcap_{n=1}^\infty A_n = \emptyset$, so there is $N \in
  \mathbb N$ with $\bigcap_{n=1}^N C_n = \emptyset$ since $\mathcal C$
  is a compact system. So, for any $m>N$, we have that
  \begin{align*}
     \mu(A_m) & = \mu\Big( \Big(\bigcap_{n=1}^m A_n\Big) \setminus
     \Big(\bigcap_{n=1}^m C_n\Big)\Big) \leq \sum_{n=1}^m \mu(A_n
     \setminus C_n) \\ & \leq \sum_{n=1}^m \delta_n < \varepsilon.
  \end{align*}
  This concludes the proof.
\end{proof}

%% def closedCompactCylinders :
%%   Set (Set ((i : ι) → α i)) :=
%%   ⋃ (s) (S) (_ : IsClosed S) (_ : IsCompact S), {cylinder s S}

%%     theorem isCompactSystem_cylinders :
%% IsCompactSystem (fun t ↦ t ∈ closedCompactCylinders α)

%% lemma AddContent.sigma_additive_of_regular [Nonempty α] (hR : SetRing R) (m : AddContent R)
%%     (hm_ne_top : ∀ {s} (_ : s ∈ R), m s ≠ ∞)
%%     (hC : IsCompactSystem C) (hCR : C ⊆ R)
%%     (h_reg : ∀ A (_ : A ∈ R) (ε : ℝ≥0∞) (_ : 0 < ε), ∃ K ∈ C, K ⊆ A ∧ m (A \ K) ≤ ε)
%%     ⦃f : ℕ → Set α⦄ (hf : ∀ i, f i ∈ R) (hUf : (⋃ i, f i) ∈ R) (h_disj : Pairwise (Disjoint on f)) :
%%     m (⋃ i, f i) = ∑' i, m (f i) 
    
%In our formalization, we are less general and use a direct argument
%for a \leanlinecolor{kolContent}. More precisely, we show that a
%\leanlinecolor{kolContent}, which is based on a projective family
%which is inner regular with respect to compact and closed sets is
%$\sigma$-additive.

In order to apply this result and show the extension theorem, we need
to show that
%$\{\pi_J^{-1} C: C \in \prod_{j\in J} \alpha_j \text{ compact}\}$ as
%  well as
$\{\pi_J^{-1} C: C \in \prod_{j\in J} \alpha_j \text{ compact and
  closed}\}$ is a compact systems. Note that compact sets are closed
in Hausdorff spaces, but we do not have this property since we are
working with pseudo-metric spaces. Since Lemma~\ref{l:stetigcompact}
gives the $\sigma$-additivity of a \leanlinecolor{kolContent}, which
is defined through the projective family \leanline{P}, we have:

\begin{minted}[highlightlines={1}, mathescape, numbersep=5pt, framesep=5mm, bgcolor=mygray]{Lean}
theorem kolContent_sigma_additive_of_innerRegular
    (hP : IsProjectiveMeasureFamily P)
    (hP_inner : ∀ J,
      (P J).InnerRegular (fun s => IsCompact s ∧ IsClosed s)
      MeasurableSet)
    ⦃f : ℕ → Set (∀ i, α i)⦄ (hf : ∀ i, f i ∈ cylinders α)
    (hf_Union : (⋃ i, f i) ∈ cylinders α)
    (h_disj : Pairwise (Disjoint on f)) :
    kolContent hP (⋃ i, f i) = ∑' i, kolContent hP (f i)
\end{minted}

Note that the assumption \leanline{hP_inner} from above translates
directly to inner regularity of the \leanlinecolor{kolContent}, which
is defined through the projective family \leanline{P}. Moreover, since
the Carathéodory extension theorem requires $\sigma$-subadditivity
(rather than $\sigma$-additivity), we can use
\leanlinecolor{sigma_subadditive_of_sigma_additive} in order
to show
\leanlinecolor{kolContent_sigma_subadditive_of_innerRegular}, and
the assumptions of the last result also imply $\sigma$-subadditivity.

%  theorem kolContent_sigma_subadditive_of_innerRegular
%    (hP : IsProjectiveMeasureFamily P)
%    (hP_inner : ∀ J,
%      (P J).InnerRegular (fun s => IsCompact s ∧ IsClosed s)
%      MeasurableSet)
%    ⦃f : ℕ → Set (∀ i, α i)⦄
%    (hf : ∀ i, f i ∈ cylinders α)
%    (hf_Union : (⋃ i, f i) ∈ cylinders α) :
%    kolContent hP (⋃ i, f i) ≤ ∑' i, kolContent hP (f i)



What remains to be done is to show conditions under which the
projective family is inner regular with respect to compact (and
closed) sets. For this, we need some assumption on the underlying
spaces. The following lemma provides the generalizations to extended
pseudo-metric spaces:

The next result is already implemented in \leanline{mathlib}: Use
\leanline{Measure.InnerRegular.of_pseudoEMetricSpace} in order to show
that every open set is inner regular with respect to closed sets, and
\leanline{Measure.InnerRegular.measurableSet_of_open} to show that
every measurable set is inner regular with respect to closed sets.
The claimed outer regularity is part of
\leanline{MeasureTheory.Measure.InnerRegular.weaklyRegular_of_finite}.

\begin{lemma}\label{l:pseude1}
  Let $r$ be an extended pseudo-metric on $\alpha$, and the topology
  on $\alpha$ be given by $r$. If $\mu$ is a finite measure on the
  Borel $\sigma$-algebra $\mathcal B$ on $\alpha$, it is inner regular
  with respect to closed sets. In fact, we have also outer regularity
  with respect to open sets, i.e.\ for all $A\in \mathcal B$,
  $$ \mu(A) = \inf \{ \mu(O) : A \subseteq O \text{ open}\}.$$
\end{lemma}

\begin{proof}
  It suffices to prove that
  $$ \mathcal A := \Big\{A \in \mathcal B : \mu(B) = \sup_{F\subseteq
    B \text{ closed}} \mu(F) = \sup_{B\subseteq O \text{ open}}
  \mu(O)\Big\} $$ is a $\sigma$-algebra containing all closed
  sets. So, we proceed in two steps.  \\ {\em Step 1: $\mathcal A$ is
    a $\sigma$-algebra:} Since $\mu$ is finite, we have that $\mu(A^c)
  = \mu(\alpha) - \mu(A)$ for all $A$. Using this, we already have
  that $\mathcal A$ is closed under complements. So, we are left with
  showing that $\mathcal A$ is closed under countable unions. For
  this, let $A_1, A_2,... \in \mathcal A$ and $A :=
  \bigcup_{n=1}^\infty A_n$. Fix $\varepsilon > 0$ and a sequence
  $\delta_1, \delta_2,... > 0$ with $\sum_{n=1}^\infty \delta_n <
  \varepsilon$. For each $n$, let $F_n \subseteq A_n \subseteq O_n$
  with $F_n$ closed, $O_n$ open and $\mu(O_n \setminus F_n) \leq
  \delta_n$. Then, $O := \bigcup_{n=1}^\infty O_n$ is open and
  $$ 0 \leq \mu(O) - \mu(A) \leq \sum_{n=1}^\infty \mu(O_n \setminus
  A) \leq \sum_{n=1}^\infty \mu(O_n \setminus A_n) < \varepsilon.$$
  This shows outer regularity of $\mu$ with respect to open sets at
  $A$. It remains to show inner regularity with respect to closed
  sets. For this, note that $\bigcup_{n=1}^N F_n$ is closed for all
  $N$ and recall from Proposition~\ref{P:stetigmass} that $\mu$ is
  continuous from below. Therefore, setting $F := \bigcup_{n=1}^\infty
  F_n$,
  \begin{align*}
    0 & \leq \mu(A) - \lim_{N\to\infty} \mu\Big(\bigcup_{n=1}^N
    F_n\Big) = \mu(A \setminus F) \leq \sum_{n=1}^\infty \mu(A_n
    \setminus F) \\ & \leq \sum_{n=1}^\infty \mu(A_n \setminus F_n) <
    \varepsilon.
  \end{align*}
  This shows inner regularity of $\mu$ with respect to closed sets at
  $A$.


  \noindent
  {\em Step 2: $\mathcal A$ contains all closed sets:} Let
  $\varepsilon_ n\downarrow 0$ and, for $\varepsilon>0$ and some $A
  \subseteq \alpha$, $A^\varepsilon := \{x : \exists y \in A,
  r(x,y)<\varepsilon\}$. Then, for $A$ closed, we have that $A =
  \bigcap_{n=1}^\infty A^{\varepsilon_n}$ and $A^{\varepsilon_n}$ is
  open for all $n$. Clearly, $\mu(A) = \sup\{\mu(F): F\text{ closed},
  F \subseteq A\}$, since $A$ is closed. By continuity of $\mu$, we
  find that $\mu(A^{\varepsilon_n}) \xrightarrow{n\to\infty} \mu(A)$,
  therefore $\mu(A) = \inf\{\mu(O): F\text{ open}, A \subseteq O\}$,
  and we have shown that $A \in \mathcal A$.
\end{proof}

%xxx From Remy: For \leanline{inner_regular_etc} we already do it in
%three parts: our result shows \leanline{P.inner_regular (λ s,
%is_compact s ∧ is_closed s) is_closed}, then we go from closed to
%open sets thanks to
%\leanline{docs3#measure_theory.measure.inner_regular.of_pseudo_emetric_space},
%and finally to measurable sets with
%\leanline{docs3#measure_theory.measure.inner_regular.measurable_set_of_open}. That
%last step uses the borel space assumption in the typeclass inference
%process to get its \leanline{outer_regular} argument.

In the next lemma, we will need that a closed subset of a compact set
is compact. (See \leanline{isCompact_of_isClosed_subset}.) We describe
the formalization of the next lemma only in conjunction with
Lemma~\ref{L:relcoPol}.

\begin{lemma}\label{l:tight}
  Let $r$ be an extended pseudo-metric on $\alpha$, and the topology
  on $\alpha$ be given by $r$. If $\mu$ is a finite measure on the
  Borel $\sigma$-algebra $\mathcal B$ on $\alpha$, the following are
  equivalent:
  \begin{enumerate}
    \item For all $\varepsilon>0$, there is some closed and compact
      $K$ with $\mu(K^c) < \varepsilon$.
    \item For all $\varepsilon>0$ and $A \in \mathcal B$, there is
      some closed and compact $K\subseteq A$ with $\mu(A \setminus K)
      < \varepsilon$.
  \end{enumerate}
\end{lemma}

\todo{too much detail? omit the proof?}

\begin{proof}
  2. $\Rightarrow$ 1.: This is clear since $\alpha \in \mathcal B$.
  1. $\Rightarrow$ 2.: Fix $\varepsilon>0$ and $A \in \mathcal B$. Let
  $K$ be closed and compact with $\mu(K^c) < \varepsilon/2$. By
  Lemma~\ref{l:pseude1}, there is some closed $F \subseteq A$
  with $\mu(A) < \mu(F) + \varepsilon/2$. Now, we have that $F \cap K
  \subseteq A$ is closed and compact, and with $A \setminus (F\cap K)
  = (A \setminus F) \cup (A \setminus K)$,
  $$\mu(A \setminus (F\cap K)) \leq \mu(A \setminus F) + \mu(A
  \setminus K) < \varepsilon/2 + \mu(K^c) < \varepsilon. $$
\end{proof}

We apply this result by using that 1.\ is satisfied for complete and
separable extended pseudo-metric spaces. Here, for the generalization
to extended pseudo-metric spaces, we use that any subset of such a
space is compact iff it is complete and totally
bounded\footnote{\label{note:tot}A subset of a pseudo-metric space is
  totally bounded iff, for all $\varepsilon>0$, it can be covered by
  finitely many balls of radius $\epsilon$.} (see
\leanline{isCompact_iff_totallyBounded_isComplete} in {\tt
  mathlib}). Since closed subsets of a complete space are complete,
the closure of a totally bounded set is (still totally bounded, hence)
compact. (This fact is used in the proof below.)

\begin{lemma}\label{L:relcoPol}
  Let $\alpha$ be a complete and separable, extended pseudo-metric
  space, and $\mu$ a finite measure on its Borel $\sigma$-algebra
  $\mathcal B$. Then, for any $\varepsilon>0$, there is some
  $K\subseteq \alpha$ with compact closure and\footnote{We write $\bar
    K$ for the closure of $K$.}  $\mu((\bar K)^c)<\varepsilon$.
\end{lemma}

\begin{proof}
 Since $\alpha$ is separable, there is a subset $\{x_1, x_2,...\}
 \subseteq_c \alpha$ with closure $\alpha$. By separability, for each
 $n\in \mathbb N$, $\alpha = \bigcup_{k=1}^\infty
 B_{\varepsilon_n}(x_k)$. Since $\mu$ is continuous from above in
 $\emptyset$ (see Lemma~\ref{P:stetigmass})
  \[
  0 = \mu\Big(\Big(\bigcup_{k=1}^\infty B_{\varepsilon_n}(x_k)\Big)^c
  \Big) = \lim_{N\to\infty} \mu\Big( \Big(\bigcup_{k=1}^N
  B_{\varepsilon_n}(x_k)\Big)^c\Big).
  \]
  Let $\delta_n\downarrow 0$ be summable with $\sum_{n=1}^\infty
  \delta_n = 1$ (e.g.\ $\delta_n = 2^{-n}$). Then, there is some
  $N_n\in\mathbb N$ with $\mu\Big( E\setminus \bigcup_{k=1}^{N_n}
  B_{\varepsilon_n}(x_k)\Big) < \varepsilon \delta_n$. Now take
  \[
    A := \bigcap_{n=1}^\infty \bigcup_{k=1}^{N_n}
    B_{\varepsilon_n}(x_k),
  \]
  which by construction is totally bounded (for $r>0$ take $n$ such
  that $\varepsilon_n < r$, such that $B_r(x_1),...,B_r(x_{N_n})$
  cover $A$), and therefore has a compact closure as noted
  in the lines directly preceding the lemma. In
  addition, by $\sigma$-sub-additivity of $\mu$,
  \begin{align*}
     \mu((\overline A)^c) & \leq \mu(A^c) = \mu\Big(
     \bigcup_{n=1}^\infty \Big(\Big(\bigcup_{k=1}^{N_n}
     B_{\varepsilon_n}(x_k)\Big)^c\Big)\Big) \\ & \leq
     \sum_{n=1}^\infty \mu\Big( \Big(\bigcup_{k=1}^{N_n}
     B_{\varepsilon_n}(x_k)\Big)^c\Big)< \varepsilon,
  \end{align*}
 and we are done.
\end{proof}

Let us combine the last two results in one combined formalized result.
This the gives the desired result that a measure on a second
countable, complete extended pseudo-metric space is inner regular with
respect to closed compact sets.

\begin{minted}[highlightlines={1}, mathescape, numbersep=5pt, framesep=5mm, bgcolor=mygray]{Lean}
theorem innerRegular_isCompact_isClosed_
    measurableSet_of_complete_countable
    [PseudoEMetricSpace α] [CompleteSpace α]
    [SecondCountableTopology α]
    [BorelSpace α] (P : Measure α) [IsFiniteMeasure P] :
    P.InnerRegular (fun s => IsCompact s ∧ IsClosed s)
      MeasurableSet 
\end{minted}

%\begin{lemma}\label{l:totalBoundcompact}
%  Let $E$ be a uniform topologocal space\footnote{This is more general
%    than a metric space, i.e.\ every metric space is uniform.} Then,
%  $A\subseteq E$ is compact iff it is totally bounded\footnote{In a
%    metric space, this means that forall $\varepsilon>0$ you find a
%    finite number of balls covering the set} and
%  complete\footnote{i.e.,\ Cauchy filters converge.}.
%\end{lemma}

%\noindent
%The usual way to apply this result is in a Polish space. Here, $A$ is
%relatively compact iff it is totally bounded.

%\noindent
%Although, regularity of measures is covered in {\tt mathlib}, the next
%statement is not yet part of the library. Similar constructions will
%be important for Prohorov's theorem in the theory of weak convergence.
%In Lean, we can use \verb+ measure_theory.measure.inner_regular+ for
%the statement of this result. In the proof of the lemma, we will be
%using the last fact

\subsection{Proof of Kolmogorov's extension theorem}
We now describe the proof of Kolmogorov's extension theorem as well as
its formalization: For the proof of Theorem~\ref{T1}, we 
\begin{enumerate}
  \item apply Theorem~\ref{T:masseind} for the ring (hence semi-ring)
    $\mathcal A$ and the set-function $P$ as given in
    Theorem~\ref{T1};
  \item show $\sigma$-additivity (as assumed in
    Theorem~\ref{T:masseind}) of $P$ by using
    Lemma~\ref{l:stetigcompact}. For the latter, we need to show that
    $P$ is inner regular with respect to a compact system. Here, note
    that $\{\pi_J^{-1}C : C \in \prod_{j\in J} \alpha_j \text{ compact
      and closed}\}$ is a compact system and $P(\pi_J^{-1}C) =
    P_J(C)$, so we need to show that $P_J$ is inner regular with
    respect to compact and closed sets, $J\subseteq_f \iota$;
  \item use Lemma~\ref{L:relcoPol} in combination with
    Lemma~\ref{l:tight} (1.$\Rightarrow$2.) and the properties of
    completeness, separability of the underlying extended
    pseudo-metric spaces in order to see that every $P_J$ has the
    desired property of being inner regular wrt compact and closed
    sets. This concludes the proof of Theorem~\ref{T1}. 
\end{enumerate}
The formalization of this proof resembles these arguments. We leave
out all instances in the reformulation of the result and its proof
(see below Theorem~\ref{T1} for a full formulation):

\begin{minted}[highlightlines={1}, mathescape, numbersep=5pt, framesep=5mm, bgcolor=mygray]{Lean}
def projectiveLimitWithWeakestHypotheses
    (P : ∀ J : Finset ι, Measure (∀ j : J, α j))
    (hP : IsProjectiveMeasureFamily P) :
    Measure (∀ i, α i) :=
  Measure.ofAddContent setSemiringCylinders
    generateFrom_cylinders (kolContent hP)
    (kolContent_sigma_subadditive_of_innerRegular hP
      fun J => innerRegular_isCompact_isClosed_
      measurableSet_of_complete_countable (P J))    
\end{minted}
\sloppy Let us look closer at the formalized proof:
\leanlinecolor{kolContent_sigma_subadditive_of_innerRegular} has
the same hypotheses as
\leanlinecolor{kolContent_sigma_additive_of_innerRegular}, except
\leanline{h_disj}. So, the lemma in the last brackets shows
$\sigma$-sub-additivity of the content $P$ from
Theorem~\ref{T1}. Then, \leanlinecolor{generateFrom_cylinders} gives
the \leanline{MeasurableSpace} on which we define the
\leanlinecolor{addContent}.


\section{Concluding thoughts}
\label{ss:concl}
{\color{blue} The usability of our contribution must be proved by an
  example: Adding to our general theory, we have implemented the
  example of an infinite product measure. (As an application of
  \leanlinecolor{projectiveLimitWithWeakestHypotheses}, we have to use
  complete and separable extended pseudo-metric spaces, but note that
  an implementation of the Ionescu-Tulcea Theorem would give
  the same result without these prerequisites.) Here, we can build on
  \leanline{Measure.pi}, which defines a finite product measure.
  Using the same automatically inferred instances, we obtain:}
  
\begin{minted}[highlightlines={1,8}, mathescape, numbersep=5pt, framesep=5mm, bgcolor=mygray]{Lean}
def independentFamily
    (P : ∀ i, Measure (α i))
    [∀ i, IsProbabilityMeasure (P i)] : Measure (∀ i, α i) :=
  projectiveLimitWithWeakestHypotheses
    (fun J : Finset ι => Measure.subset_pi P J)
    (product_isProjective P)

def Measure.subset_pi
    (P : ∀ i, Measure (α i)) (I : Finset ι) :
    Measure (∀ i : I, α i) :=
  Measure.pi (fun (i : I) => (P (i : ι)))
\end{minted}

Moreover, we will highlight some paths we imagine can be taken,
leading to further extensions of \leanline{mathlib}, which build on
our contribution.

{\em Using semi-rings in measure theory:} Frequently, the construction
of measures on some $\sigma$-algebra $\mathcal F$ (on some space
$\alpha$) uses outer measures, which can be defined on $2^\alpha$ (the
set of all subsets of $\alpha$). We provide a general framework using
Carathéodory's extension theorem, which states an extension of a
set-function on a semi-ring by an outer measure, not previously
implemented in mathlib. This could also be used to re-define Stieltjes
measures (in particular Lebesgue measure) on $\mathbb R$, as well as
product measures (on finite products). More precisely,
e.g.\ \leanline{StieltjesFunction.outer_Ioc} and
\leanline{MeasureTheory.constructions.pi.pi_pi_aux} are specific
examples of the more general statement, that the outer measure extends
the set function a the semi-ring. (In the first case, it would be the
semi-ring of half-open intervals, in the second case, it would be the
semi-ring of cylinder sets on a product space.)

\noindent
{\em Prohorovs theorem:} The proof that single finite measures on a
Polish space is tight (i.e.\ inner regular with respect to compact
sets) is a special case of Prohorov's theorem, which states that a set
$\mathcal M$ of finite measures is relatively compact (in the weak
topology on the set of finite measures, i.e.\ every sequence of
measures in the set has a weakly convergent sub-sequence) if and only
if $\mathcal M$ is tight (i.e.\ for all $\varepsilon>0$ there is some
compact $K$ such that $\sup_{m \in \mathcal M} m(K^c) <
\varepsilon$). We treat here the special case of $\mathcal M$ being a
singleton (hence is a compact set).

~~



\section*{Acknowledgements}

\bibliographystyle{plain}
\bibliography{main}




\end{document}
